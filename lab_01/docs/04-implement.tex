\chapter{Технологический раздел}

В данном разделе будут описаны средства реализации ПО и листинга кода алгоритмов, а также рассмотрены тестовые случаи.

\section{Средства реализации} 

Для реализации программ я выбрал язык программирования Python, так как я очень хорошо знаком c этим языком и пишу на нем давно, также этот язык является удобным, безопасным и в нем присутствуют инструменты замера времени.

\section{Листинги кода}

В листингах 3.1 - 3.4 приведены реализации алгоритмов, изученных в аналитическом разделе.

\captionsetup{singlelinecheck = false, justification=raggedright}
\begin{lstlisting}[label=recLevenshtain, caption=Рекурсивный Левенштейн]
def recLevenshtain(left_str: str, right_str: str) -> int:
    if left_str == "" or right_str == "":
        return abs(len(left_str) - len(right_str))

    if left_str[-1] == right_str[-1]:
        c = 0
    else:
        c = 1

    return min(
        recLevenshtain(left_str, right_str[:-1]) + 1,
        recLevenshtain(left_str[:-1], right_str) + 1,
        recLevenshtain(left_str[:-1], right_str[:-1]) + c,
    )
\end{lstlisting}

\captionsetup{singlelinecheck = false, justification=raggedright}
\begin{lstlisting}[label=cacheLevenshtain, caption=Кеширующий Левенштейн]
def cacheLevenshtain(left_str: str, right_str: str) -> int:
    left_str_len = len(left_str) + 1
    right_str_len = len(right_str) + 1
    matrix = [[i + j for j in range(right_str_len)] \
                        for i in range(left_str_len)]

    for i in range(1, left_str_len):
        for j in range(1, right_str_len):
            if left_str[i - 1] == right_str[j - 1]:
                c = 0
            else:
                c = 1

            matrix[i][j] = min(
                matrix[i - 1][j] + 1, matrix[i][j - 1] + 1,\
                matrix[i - 1][j - 1] + c
            )

    return matrix[-1][-1]
\end{lstlisting}

\captionsetup{singlelinecheck = false, justification=raggedright}
\begin{lstlisting}[label=recDamerayLevenshtain, caption=Рекурсивный Дамерау-Левенштейн]
def recDamerayLevenshtain(left_str: str, right_str: str) -> int:
    if left_str == "" or right_str == "":
        return abs(len(left_str) - len(right_str))
        
    if left_str[-1] == right_str[-1]:
        c = 0
    else:
        c = 1

    result = min(
        recDamerayLevenshtain(left_str, right_str[:-1]) + 1,
        recDamerayLevenshtain(left_str[:-1], right_str) + 1,
        recDamerayLevenshtain(left_str[:-1], right_str[:-1]) + c,
    )

    if (
        len(left_str) >= 2
        and len(right_str) >= 2
        and left_str[-1] == right_str[-2]
        and left_str[-2] == right_str[-1]
    ):
        result = min(result, recDamerayLevenshtain(left_str[:-2], \
                                                right_str[:-2]) + 1)

    return result
\end{lstlisting}

\captionsetup{singlelinecheck = false, justification=raggedright}
\begin{lstlisting} [label=cacheDamerayLevenshtain, caption=Кеширующий Дамерау-Левенштейн]
def cacheDamerayLevenshtain(left_str: str, right_str: str) -> int:
    left_str_len = len(left_str) + 1
    right_str_len = len(right_str) + 1
    matrix = [[i + j for j in range(right_str_len)]\
                      for i in range(left_str_len)]

    for i in range(1, left_str_len):
        for j in range(1, right_str_len):
            if left_str[i - 1] == right_str[j - 1]:
                c = 0
            else:
                c = 1

            matrix[i][j] = min(
                matrix[i - 1][j] + 1, matrix[i][j - 1] + 1,\
                matrix[i - 1][j - 1] + c
            )

            if (
                (i > 1 and j > 1)
                and left_str[i - 1] == right_str[j - 2]
                and left_str[i - 2] == right_str[j - 1]
            ):
                matrix[i][j] = min(matrix[i][j], matrix[i - 2][j - 2] + 1)

    return matrix[-1][-1]
\end{lstlisting}

\section{Тестирование ПО}

В таблице 3.1 приведены тестовые случаи для алгоритмов поиска редакционного расстояния.

\begin{table}[H]
	\begin{center}
    	\captionsetup{singlelinecheck = false, justification=centering}
		\caption{Тестовые случаи}
		\begin{tabular}{c|c|c|c|c}
			№ & Строка 1 & Строка 2 & Левенштейн & Д.-Левенштейн\\
			\hline
			1 & кит & кот & 1 & 1\\
			2 & рома & роам & 2 & 1\\
			3 & чикипики & чикипик & 1 & 1\\
			4 & сальто & сльто & 1 & 1\\
			5 & "" & kekw & 4 & 4\\
		\end{tabular}
	\end{center}
\end{table}

\section{Вывод}

В данном разделе были представлены выбор языка программирования, листинги реализаций алгоритмов и результаты тестирования.