\setcounter{page}{4}
\addchap{Введение}

В настоящее время перед компьютерной лингвистикой ставится множество задач. Одна из них - поиск редакционного расстояния между строками. Это определение минимального количества редакционных операций, необходимых для превращения одной строки в другую. Впервые эту задачу обозначил В. И. Левенштейн, имя которого закрепилось за ней.

При вычислении расстояния Левенштейна редакционные операции ограничиваются вставкой, удалением и заменой. В случае расстояния Дамерау - Левенштейна к операциям добавляется транспозиция - перестановка двух соседних символов.

Данные алгоритмы находят применение не только в компьютерной лингвистике для исправления ошибок или автозамены слов, но также в биоинформатике для определения разных участков ДНК и РНК.

Существует множество модификаций упомянутых алгоритмов. В данной работе будут рассмотрены лишь те, которые используют парадигмы динамического программирования.

Целью данной работы является реализация и изучение следующих алгоритмов нахождения редакционного расстояния:

\begin{itemize}
    \item рекурсивный Левенштейн;
    \item кеширующий Левенштейн;
    \item рекурсивный Дамерау - Левенштейна;
    \item кеширующий Дамерау - Левенштейна.
\end{itemize}

Для достижения поставленной цели необходимо выполнить следующие задачи:

\begin{itemize}
	\item изучить выбранные алгоритмы нахождения редакционного расстояния;
	\item составить схемы рассмотренных алгоритмов;
	\item реализовать разработанные алгоритмы нахождения редакционного расстояния;
	\item провести сравнительный анализ алгоритмов по затрачиваемым ресурсам (время и память);
	\item описать и обосновать полученные результаты.
\end{itemize}
